%%%%%%%%%%%%%%%%%%%%%%%%%%%%%%%%%%%%%%%%%%%%%%%%%%%%%%%%%%%%%%%%%%%%%%%%
\chapter{Magnetic Resonance Imaging} \label{chap:MagneticResonanceImaging}
%%%%%%%%%%%%%%%%%%%%%%%%%%%%%%%%%%%%%%%%%%%%%%%%%%%%%%%%%%%%%%%%%%%%%%%%
\vspace{1cm}

Magnetic resonance imaging (MRI) is one of the most powerful and versatile techniques in modern medical imaging. Its non-invasive nature and the absence of ionizing radiation have established it as a cornerstone in both clinical diagnostics and biomedical research. MRI exploits the fundamental physical interactions between nuclear spins and magnetic fields, providing not only high-resolution structural images, but also access to functional, metabolic, and microstructural information.

This chapter goes through the fundamental concepts of MRI. First, it introduces the underlying physical principles of nuclear magnetic resonance, including spin dynamics, relaxation processes, and signal generation. Second, it describes how spatial encoding is achieved through magnetic field gradients, which allow the reconstruction of three-dimensional images.

\section{Physical Principles}

\section{Spatial Encoding}
