\documentclass[12pt,a4paper]{report}
\usepackage[footskip=0.2cm]{geometry}
\usepackage[english]{babel}
\usepackage[sfdefault,lf]{carlito}
\usepackage{color}
\usepackage{setspace}
\usepackage{float}
\usepackage{graphicx} %insert figures
\usepackage{fancyhdr}
\usepackage{etoolbox}

\renewcommand{\headrulewidth}{0pt}
\renewcommand{\footrulewidth}{1pt}

\begin{document}
% geometry settings for title page 
    \newgeometry{left=20mm, top=25mm, right=15mm, bottom=35mm}
% The title page settings can be changed in titlepage.tex document
\begin{titlepage}
        \thispagestyle{fancy}
    \begin{figure}[h]
        \vspace{-0.5cm} % changes vertical spacing 
        \centering
        \includegraphics[scale=1.05]{logo_unibo.png}
    \end{figure}

    \vspace{3mm}

    \begin{center}
        \bf {DEPARTMENT OF PHYSICS AND ASTRONOMY ``A. RIGHI''}
    \end{center}

    \begin{center}
        \bf \large{SECOND CYCLE DEGREE}

        \bf \large{PHYSICS}
    \end{center}

    \vspace{15mm}

    \begin{center}
        %
        % INSERT THE TITLE OF YOUR THESIS
        %
        {\huge{\bf 3D U-NET DOMAIN GENERALIZATION IN}}
        \vspace{15pt}\\
        {\huge{\bf FETAL BRAIN MRI SEGMENTATION}}
    \end{center}

    \vspace{50mm} \par \noindent

    \begin{minipage}[t]{0.47\textwidth}
        %
        % INSERT THE NAME OF THE SUPERVISOR WITH ITS TITLE (DR. OR PROF.)
        %
        \vspace*{-19pt}
        {\large{\bf Supervisor
        \vspace{2mm}\\
        Prof.\ Daniel Remondini\\\\
        %
        % INSERT THE NAME OF THE CO-SUPERVISOR WITH ITS TITLE (DR. OR PROF.)
        %
        % IF THERE ARE NO CO-SUPERVISORS REMOVE THE FOLLOWING 5 LINES
        %
        Co-supervisors
        \vspace{2mm}\\
        Dr.\ Nico Curti\\
        Dr.\ Gerard Martí Juan\\\\}}
    \end{minipage}
    %
    \hfill
    %
    \begin{minipage}[L]{0.23\textwidth} \textcolor{black}{
        {\large{\bf  Defended by
        \vspace{2mm}\\
        \noindent{Simone Chiarella}}}
        }
    \end{minipage}
    %
    \cfoot[C]{
        \\
        \bf 18--19/12/2025\\
        \vspace{3mm}
        \bf Academic Year 2024/2025
    }

\end{titlepage}
\restoregeometry

\newpage\null\thispagestyle{empty}\newpage

\renewcommand\abstractname{\textsc{Abstract}}
{
  \fontfamily{bch}\selectfont\begin{abstract}
    This work investigates domain generalization of a 3D U-Net (nnU-Net v2.4.1, ResEncM) for fetal brain MRI segmentation across three datasets with different characteristics: Kispi-mial, Kispi-irtk, and dHCP. The causality-inspired augmentation GIN-IPA was integrated into the nnU-Net training loop, and three strategies were compared: default nnU-Net augmentation, GIN-IPA augmentation alone, and their combination. Models are trained on each dataset and evaluated both in-domain and out-of-domain using Dice score, volume similarity and Hausdorff distance. Two main conclusions emerge. First, dataset quality and scale dominate generalization: training on dHCP yields consistently stable performance across domains, largely insensitive to the augmentation recipe. Second, GIN-IPA provides gains, but only when the source lacks the target variability: with training on Kispi-irtk and inference on dHCP, all the performance metrics rise significantly. Instead, stacking default augmentation with GIN-IPA is not additive and can be detrimental. Limitations include a general small availability of public data and label-set harmonization. The results argue for prioritizing multi-center, high-quality fetal MRI with standardized SR reconstruction; within constrained settings, GIN-IPA may represent a useful and pragmatic choice for single-source domain generalization.
  \end{abstract}
}

\end{document}
