\documentclass[12pt,a4paper]{report}
\usepackage[english]{babel}
\usepackage{newlfont}
\usepackage{color}
\usepackage{setspace}
\textwidth=450pt\oddsidemargin=0pt

\begin{document}
\begin{titlepage}
%
%
% ONCE YOU ARE FINISHED WITH YOUR CHANGES MODIFY "RED" WITH "BLACK" IN ALL \textcolor COMMENTS
%
%
\begin{center}
{{\Large{\textsc{Alma Mater Studiorum $\cdot$ University of  Bologna}}}} 
\rule[0.1cm]{15.8cm}{0.1mm}
\rule[0.5cm]{15.8cm}{0.6mm}
\\\vspace{3mm}
{\small{\bfseries School of Science \\
Department of Physics and Astronomy\\
Master Degree in Physics}}
\end{center}

\vspace{23mm}

\begin{center}
%
% INSERT THE TITLE OF YOUR THESIS
%
\begin{spacing}{2}
{\LARGE{\bfseries 3D U-NET DOMAIN GENERALIZATION IN FETAL BRAIN MRI SEGMENTATION}}\\
\end{spacing}
\end{center}

\vspace{35mm} \par \noindent

\begin{minipage}[t]{0.47\textwidth}
%
% INSERT THE NAME OF THE SUPERVISOR WITH ITS TITLE (DR. OR PROF.)
%
{\large{\bfseries Supervisor: \vspace{2mm}\\{
Prof.\,Daniel Remondini}\\\\
%
% INSERT THE NAME OF THE CO-SUPERVISOR WITH ITS TITLE (DR. OR PROF.)
%
% IF THERE ARE NO CO-SUPERVISORS REMOVE THE FOLLOWING 5 LINES
%
\bfseries Co-supervisors:
\vspace{2mm}\\
Dr.\,Nico Curti\\
Dr.\,Gerard Martí Juan\\\\}}
\end{minipage}
%
\hfill
%
\begin{minipage}[t]{0.47\textwidth}\raggedleft \textcolor{black}{
{\large{\bfseries Submitted by:
\vspace{2mm}\\
%
% INSERT THE NAME OF THE GRADUAND
%
Simone Chiarella}}}

\end{minipage}

\vspace{20mm}

\begin{center}
%
% INSERT THE ACADEMIC YEAR
%
Academic Year 2024/2025
\end{center}

\end{titlepage}

\newpage\null\thispagestyle{empty}\newpage

\renewcommand\abstractname{\textsc{Abstract}}
{
  \fontfamily{bch}\selectfont\begin{abstract}
    This work investigates domain generalization of a 3D U-Net (nnU-Net v2.4.1, ResEncM) for fetal brain MRI segmentation across three datasets with different characteristics: Kispi-mial, Kispi-irtk, and dHCP. The causality-inspired augmentation GIN-IPA was integrated into the nnU-Net training loop, and three strategies were compared: default nnU-Net augmentation, GIN-IPA augmentation alone, and their combination. Models are trained on each dataset and evaluated both in-domain and out-of-domain using Dice score, volume similarity and Hausdorff distance. Two main conclusions emerge. First, dataset quality and scale dominate generalization: training on dHCP yields consistently stable performance across domains, largely insensitive to the augmentation recipe. Second, GIN-IPA provides gains, but only when the source lacks the target variability: with training on Kispi-irtk and inference on dHCP, all the performance metrics rise significantly. Instead, stacking default augmentation with GIN-IPA is not additive and can be detrimental. Limitations include a general small availability of public data and label-set harmonization. The results argue for prioritizing multi-center, high-quality fetal MRI with standardized SR reconstruction; within constrained settings, GIN-IPA may represent a useful and pragmatic choice for single-source domain generalization.
  \end{abstract}
}

\end{document}
