%\newpairofpagestyles[scrheadings]{conclusions}{\ohead{Conclusions}}
%\thispagestyle{conclusions}
\addcontentsline{toc}{part}{Conclusions}

%%%%%%%%%%%%%%%%%%%%%%%%%%%%%%%%%%%%%%%%%%%%%%%%%%%%%%%%%%%%%%%%%%%%%%%%
\chapter*{Conclusions}
%%%%%%%%%%%%%%%%%%%%%%%%%%%%%%%%%%%%%%%%%%%%%%%%%%%%%%%%%%%%%%%%%%%%%%%%
\vspace{1cm}

This thesis examined whether a 3D U-Net can sustain cross-domain performance in fetal brain MRI segmentation through causality-inspired data augmentation (GIN-IPA), and whether stacking it with the standard nnU-Net augmentation yields additional gains. The most important result is that the quality and coverage of the training data dominate domain generalization: models trained on a large, clean, and internally consistent source generalize more stably than models trained on smaller or noisier sources, largely independent of the augmentation recipe. Augmentation matters most when the source domain is limited and distributionally distant from the target.

The central evidence for the validation of GIN-IPA comes from the cross-domain transfer where models trained on Kispi-irtk are evaluated on dHCP. In this setting, GIN-IPA \emph{substantially improves} accuracy and robustness over the baseline, with effects that are both practically relevant and statistically significant across tissues and metrics---especially deep gray matter, ventricles and brainstem. These trends are confirmed by Wilcoxon signed-rank tests with \textit{p}-values smaller than \num{0.01}. When trained on dHCP, by contrast, augmentation choices have \emph{marginal impact}, underscoring that data fidelity and reconstruction consistency are first-order factors for generalization.

A second conclusion is that stacking heterogeneous transformations is not beneficial at all. The combined augmentation is not superior to GIN-IPA alone and, in the aforementioned most informative transfer, it is consistently inferior for several structures and metrics. This non-additivity suggests interference between transformation families that may dilute the invariances that GIN-IPA aims to enforce. As expected, more transformations do not automatically translate into better OOD behavior.

As a side result, the analysis stratified by pathology indicates that global domain shifts remain the predominant driver of performance variability. Differences between neurotypical and pathological cases were present but moderate, in part due to the different scan quality of the subcohorts. Gains from GIN-IPA are only noticeable among pathological scans, albeit with small margins, which is consistent with the expectation that morphology-induced variability is only partially addressable by appearance-focused perturbations.

The broader implication is methodological and organizational. For prenatal neuroimaging pipelines that must transfer across sites and reconstruction stacks, investment should prioritize data curation, SRR standardization, and label governance before augmentation engineering. Where multi-center high-quality data are unavailable, GIN-IPA offers an effective, lightweight mechanism to mitigate acquisition-driven shifts. Conversely, when ample high-quality data are available, the return on complex augmentation schedules is limited; rigorous cross-domain validation and careful dataset assembly are the factors that matter most.

In summary, domain generalization in fetal brain MRI is primarily determined by the data pathway, with GIN-IPA yielding substantial benefits exactly when the source domain is narrow or mismatched to the target---although its effectiveness would need further validation. These findings translate into clear guidance for building robust, transferable segmentation systems: prioritize data, deploy targeted augmentation where it closes the domain gap, and evaluate decisions on out-of-domain tests. Future work should disentangle the contributions of the two components within GIN-IPA, and validate it on larger, more diverse datasets.
