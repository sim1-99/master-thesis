%%%%%%%%%%%%%%%%%%%%%%%%%%%%%%%%%%%%%%%%%%%%%%%%%%%%%%%%%%%%%%%%%%%%%%%%
\chapter{Fetal Brain MRI} \label{chap:FetalBrainMRI}
%%%%%%%%%%%%%%%%%%%%%%%%%%%%%%%%%%%%%%%%%%%%%%%%%%%%%%%%%%%%%%%%%%%%%%%%
\vspace{1cm}



\section{Fetal Brain Structures}

\section{Acquisition Protocols}

\section{Parcellation Protocols}

\section{Super Resolution Reconstruction}

\section{Data Specifications}

\section{Challenges} \label{sec:Challenges}
MRI, and in particular fetal brain MRI, presents some characteristics that contribute to significant domain shifts, compromising the performance of deep learning models on unseen data distributions\,\cite{FeTA2024_paper, FeTA2024_review}.

On the biological aspect, the fetal brain is very challenging \textit{per se}, since it undergoes rapid and complex changes throughout gestation:
\begin{description}
    \item[Rapid Morphological Evolution] The brain structure and appearance reorganize dramatically during prenatal development\,\cite{FeTA2024_paper, Ciceri2023}. Defining and consistently identifying different brain structures across varying gestational ages is difficult due to ongoing neuronal migration, gyrification, and sulcation patterns. This means that images acquired at different GAs constitute distinct sub-domains, making a single model challenging to generalize across the entire gestational spectrum\,\cite{Ciceri2024}.
    \item[Low Tissue Contrast] The fetal brain shows different tissue contrast compared to postnatal brains. Specifically, the intensity difference between white matter and gray matter is reduced due to the absence of myelin in the fetal WM. This causes WM to appear brighter than GM on T2w images\,\cite{Ciceri2023}. Tissue contrast changes are also significantly influenced by gestational age\,\cite{FeTA2024_paper}.
    \item[Pathological Heterogeneity] Congenital disorders introduce further significant morphological variations. Training segmentation algorithms exclusively on neurotypical samples can reduce their robustness when encountering altered morphologies typical of pathological cases (e.g., spina bifida, ventriculomegaly, corpus callosum malformations). The rarity and wide variability of fetal pathologies contribute to the challenge of obtaining sufficient data for exhaustive datasets\,\cite{FeTA2024_paper, FeTA2024_review}.
\end{description}

On the technical side, the nature of in-utero MRI acquisition introduces further limitations\,\cite{Ciceri2023}:
\begin{description}
    \item[Motion Artifacts] Spontaneous fetal movement and maternal breathing during MRI acquisition lead to various artifacts, such as in-plane image blur, slice crosstalk and incongruence in slice location, significantly degrading the quality of individual slices and the reconstructed volume. While ultrafast 2D sequences (e.g., single-shot fast spin-echo, \textsc{haste}) are commonly employed to minimize these effects by acquiring each slice rapidly, they cannot entirely eliminate motion-related issues.
    \item[Low Contrast-to-Noise Ratio] The small size of the fetal brain and the need for shorter scanning periods to minimize motion contribute to a low CNR. This, coupled with acquisition limits such as thick slices---to achieve good SNR---, makes it difficult to distinguish fine anatomical details.
    \item[Intensity Inhomogeneities] Radiofrequency field inhomogeneity, non-uniform reception coil sensitivity, eddy currents driven by field gradients, and electromagnetic interactions with the body can cause intensity inhomogeneities. Such variations produce intensity bias fields across the image space, making consistent tissue intensity representation challenging for automated methods.
    \item[Partial Volume Effects] Due to thick slices, a single voxel may contain signals from multiple tissue types. This \enquote{mixing} of signals leads to ambiguous boundaries and can result in mislabeled segmentations, especially at tissue interfaces. For instance, the mixing of the CSF and cortical GM boundary leads to intensities similar to the intensity profile of the WM.
\end{description}

Last but not least, the limited availability of large, standardized, and annotated datasets further jeopardizes the development of robust segmentation algorithms. This is due to\,\cite{FeTA2024_paper,FeTA2021_review, Ciceri2023}:
\begin{description}
    \item[Small and Heterogeneous Cohorts] Fetal MRI studies often require collaboration from specialized clinical centers due to the small and vulnerable patient populations. This leads to small datasets at individual institutions, making it difficult to create uniform in-house single-center datasets.
    \item[Variability in Acquisition Protocols and Hardware] Data from different clinical centers often exhibit considerable variability in image acquisition parameters, MRI scanner hardware, and manufacturer specifications. The most important differences use to be in magnetic field strengths (e.g., 0.55\,T, 1.5\,T, 3\,T) and specific sequence parameters (e.g., TR/TE values, flip angles, FOV, slice thickness). The associated acquisition shift cause deep learning models trained on one domain to perform poorly on data from another.
    \item[SR Reconstruction Variability] Fetal MR images are typically acquired as orthogonal stacks of 2D slices to mitigate motion, which then require SR reconstruction algorithms to generate high-resolution 3D volumes. The use of different SRR algorithms (e.g., \textsc{mialsrtk}\,\cite{Tourbier2015, MIALSRTK}, \textsc{irtk}\,\cite{Kuklisova2012, irtk-simple}, \textsc{NiftyMIC}\,\cite{Ebner2020}, \textsc{svrtk}\,\cite{Uus2022}) introduces an additional layer of domain shift.
    \item[Manual Annotation Variability] Creating high-quality, manually labeled data is time-consuming---often taking several hours per case---and requires expert anatomical knowledge. This process is also prone to human error and significant inter-rater variability.
\end{description}

In conclusion, the multifaceted nature of these challenges makes robust and generalizable fetal brain MRI segmentation an exceptionally difficult task. Addressing these domain shifts is paramount for the development of deep learning models aimed to be used in clinical practice for prenatal diagnosis and research.
