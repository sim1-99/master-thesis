%%%%%%%%%%%%%%%%%%%%%%%%%%%%%%%%%%%%%%%%%%%%%%%%%%%%%%%%%%%%%%%%%%%%%%%%
\chapter{Fetal Brain MRI} \label{chap:FetalBrainMRI}
%%%%%%%%%%%%%%%%%%%%%%%%%%%%%%%%%%%%%%%%%%%%%%%%%%%%%%%%%%%%%%%%%%%%%%%%
\vspace{1cm}

MRI is an indispensable tool for the study of the developing human brain, thanks to its non-invasive nature. In the prenatal context, MRI enables the assessment of brain structures that are critical for monitoring neurodevelopment and detecting abnormalities, making fetal brain MRI a cornerstone in research settings and, when malformations are suspected, in clinical practice.

The developing brain presents features that evolve rapidly throughout gestation, requiring imaging protocols tailored to capture fine structural details while minimizing motion-related artifacts. Standardized acquisition protocols, combined with advanced post-processing techniques such as super-resolution reconstruction, are essential to obtaining images of sufficient quality for clinical diagnosis and quantitative analysis. Additionally, adequate parcellation of fetal brain structures is fundamental for studying developmental trajectories and for training automated segmentation models.

Despite these advances, fetal brain MRI remains technically challenging. Factors such as spontaneous fetal motion, maternal respiration, and variability in scanner hardware or acquisition settings introduce significant heterogeneity in the acquired data. This variability has implications not only for clinical interpretation but also for the development of automated tools.

The present chapter provides an overview of the aforementioned key aspects, presenting the main brain structures of interest during fetal development, alongside typical acquisition and parcellation protocols. Then, the most popular and effective super-resolution reconstruction algorithms are discussed. Finally, it summarizes the current challenges in the field of automated fetal brain segmentation.

\section{Fetal Brain Structures}
Here the fetal brain structures considered in this study are described, focusing on their function and developmental trajectory during gestation.
\begin{description}
    \item[Cerebrospinal fluid (CSF)] CSF mechanically protects the brain, mediates solute transport, and provides a regulated chemical environment for neurodevelopment. In the fetus, the choroid plexus is the dominant CSF-secretory epithelium. It forms early within the ventricular system; directional CSF flow from the lateral ventricles through third and fourth ventricles is established during mid-gestation. The blood-CSF barrier properties of the plexus epithelia mature prenatally, and CSF composition changes over gestation as neurogenesis declines after the 27\th gestational week (GW)\,\cite{Lun2015}.
    \item[Cortical gray matter (cGM)] cGM contains the neuronal bodies and local circuits that will support cortical computation. At fetal stages it is a forming laminar sheet, called cortical plate. From the 15\th GW the plate thickens and transitions toward a recognizable six-layer pattern across the preterm window---from the 26\th to the 36\th GW. These structural reorganizations drive age-dependent MRI contrast in cGM and its interfaces with WM and CSF.
    \item[White matter (WM)] White matter aggregates developing long-range axonal pathways that support inter-areal communication. In the fetus, the WM corresponds largely to the intermediate zone and periventricular crossroads, where major tracts traverse before compact myelination. Substantial myelination is minimal in utero and accelerates around the 10\th fetal month and postnatally, which explains the limited T1/T2 shortening of WM in fetal scans\,\cite{Vasung2019, Wilson2021}.
    \item[Ventricles] The lateral, third, and fourth ventricles are the ducts of CSF and serve as robust anatomical landmarks for orienting fetal images. Their size and shape are clinically relevant, for instance in ventriculomegaly. The ventricular system is established early as the telencephalon expands; during mid-gestation the lateral ventricles are proportionally prominent, with progressive reduction of relative size as parenchyma (cortical plate and WM) expands. Choroid plexus maturation within the ventricles advances side by side with CSF functional maturation\,\cite{Kostovic2019, Lun2015}.
    \item[Cerebellum] The cerebellum coordinates motor processing. In the fetus it undergoes rapid volumetric growth and lobulation that are visible on T2-weighted MRI and provide robust posterior fossa landmarks. Cerebellar primordia arise early; the volume growth is more consistent after the 20\th GW, with accelerated growth in the third trimester. After the 30\th GW, cerebellar growth outpaces brainstem growth. These dynamics are crucial for segmentation because the cerebellar cortex exhibits layered signal changes across late gestation\,\cite{Scott2012}.
    \item[Deep gray matter (dGM)] dGM includes thalamus and basal ganglia, which convey cortical information. Their maturation influences the timing of the cortical input to cGM. Thalamus differentiates early and sends axons to the cortical plate around the late second trimester\,\cite{Kostovic2019}.
    \item[Brainstem (BS)] The brainstem houses vital autonomic and motor functions and ascending/descending tracts. It anchors long-range connectivity to the forebrain and the cerebellum. Quantitative \textit{in vivo} fetal MRI shows specific spatiotemporal growth, with relatively faster expansion before the 30\th GW and slower changes thereafter, in contrast to the cerebellum. These coordinated but offset trajectories shape the relative contrast and morphology of the posterior fossa labels in fetal MRI\,\cite{Dovjak2021}.
\end{description}
From the development of the aforementioned structures, three generic trends can be recognized, that impact label separability on T2-weighted fetal MRI\,\cite{Vasung2019}:
\begin{itemize}
    \item The laminar reorganization of the telencephalic wall---including the cortical plate thickening---alters cGM-WM boundaries.
    \item The limited prenatal myelination maintains WM relatively T2-hyperintense compared to postnatal scans.
    \item The posterior fossa growth asynchrony---brainstem vs. cerebellum---changes the local curvature and partial-volume patterns.
\end{itemize}

\section{Acquisition Protocols}
First of all, it must be said that fetal MRI is not a screening tool, but rather a powerful, case-specific examination, when malformations are suspected. It complements ultrasonography and should be tailored to a focused diagnostic question. Protocols must optimize contrast, spatial resolution, and temporal efficiency under strict safety constraints and frequent fetal motion.

The magnetic field strength is a crucial determinant of both signal-to-noise ratio (SNR) and artifact behavior. Historically, most fetal MRI studies have been performed at \qty{1.5}{\tesla}, which ensures stable image quality with limited dielectric and susceptibility effects. Recent technical advances, however, have enabled the transition toward \qty{3}{\tesla} scanners, providing an increase in SNR that can be traded for higher spatial resolution or shorter acquisition times\,\cite{Manganaro2023,Victoria2016}.

At \qty{3}{\tesla}, dielectric artifacts, field inhomogeneities, and chemical shift distortions become more pronounced, especially in large maternal abdomens with high amniotic-fluid content. Specific absorption rate (SAR) also increases with the square of the field strength, imposing strict limits on radiofrequency power deposition. Nevertheless, multiple investigations have confirmed the absence of fetal growth retardation or auditory damage under clinically approved exposure conditions\,\cite{Chartier2019,Jaimes2019}. The European Society of Paediatric Radiology recommends preferential use of \qty{3}{\tesla} for neurological and small-structure indications, particularly those involving the posterior fossa or parenchymal lesions, while \qty{1.5}{\tesla} remains preferable in cases of polyhydramnios or when fluid-related effects significantly degrade image quality\,\cite{Manganaro2023,Colleran2022}.

Recent technological developments have revived the use of low-field MRI, particularly at \qty{0.55}{\tesla}, as a promising alternative for fetal imaging. Compared to conventional \qty{1.5}{\tesla} and \qty{3}{\tesla} systems, low-field scanners provide markedly improved magnetic field and radiofrequency pulse homogeneity, longer $T_2^*$ relaxation times, and substantially reduced SAR and acoustic noise, thereby improving maternal comfort\,\cite{Ponrartana2023}. The lower field strength also reduces the occurrence of artifacts and enables the use of higher flip angles and longer echo trains without exceeding SAR limits. For instance, \textsc{haste} sequences can employ \qty{180}{\degree} refocusing pulses and bSSFP acquisitions can adopt contrast-optimal flip angles of approximately \qty{120}{\degree}, while it can reach only \qty{60}{\degree} in a \qty{1.5}{\tesla} setting\,\cite{Ponrartana2023}. 

Despite the lower polarization and signal, these limitations are largely compensated by optimized sequence design and the intrinsic relaxometric properties of tissues at low field. The larger bore diameter---\num{70}-\num{80}\,cm---and reduced infrastructure requirements further extend accessibility to smaller clinical centers, potentially democratizing fetal MRI worldwide\,\cite{Ponrartana2023,NevesSilva2024}. Nevertheless, the reduced SNR often necessitates thicker slices or increased averaging, and advanced reconstruction or AI-based denoising strategies are being explored to overcome this constraint\,\cite{NevesSilva2024}. In \cite{Aviles2023}, the authors have demonstrated the clinical feasibility and reliability of \qty{0.55}{\tesla} fetal MRI across gestational ages between \num{17} and \num{39} weeks, even though clear limitations remain about the small population sample and its representativeness of the clinical population.

Modern fetal MRI protocols are dominated by ultrafast T2-weighted and steady-state sequences capable of minimizing the effects of fetal motion. Typical sequence classes include single-shot fast spin-echo (\textsc{ssfse} or \textsc{haste}), balanced steady-state free precession (bSSFP), and diffusion-weighted or intravoxel incoherent motion (\textsc{ivim}) imaging.
\begin{description}
    \item[\textsc{ssfse}/\textsc{haste}] It is a free-breathing sequence, often the technique of choice in fetal MRI. Each slice is acquired within a single repetition interval, making it highly robust to bulk motion while providing strong T2-weighted contrast that delineates cerebrospinal fluid, cortical gray matter, and white matter. At \qty{3}{\tesla}, dielectric shading can be mitigated by prescan filters, flip-angle adjustment, or strategic placement of saturation bands\,\cite{Manganaro2023,Soher2007}.
    \item[bSSFP] It provides high SNR and mixed T1/T2 weighting, enhancing visualization of vascular and fluid-filled structures, including the fetal heart and umbilical cord. At higher fields, frequency-offset scouting is used to shift banding artifacts outside the region of interest. These sequences are often acquired in free-breathing mode, because temporal resolution is good enough to avoid respiratory artifacts\,\cite{Manganaro2023}.
    \item[Diffusion-weighted imaging] This class of sequences and the \textsc{ivim} protocol provide complementary information on fetal microstructure and perfusion. They enable simultaneous estimation of diffusion and perfusion parameters, that have proven valuable in characterizing placental and fetal-brain development, distinguishing between normal and growth-restricted conditions\,\cite{Ercolani2021,Antonelli2022}.
\end{description}

Besides the sequence, the design of the imaging geometry is critical for maximizing diagnostic yield and data quality for subsequent computational analysis. Fetal motion remains the major source of image degradation, despite employing ultrafast acquisitions that minimize motion sensitivity.

\section{Parcellation Protocols}
Currently, there is no established consensus in the existing fetal brain manual annotation protocols\,\cite{Uus2023}. In this field, atlases, scientific papers and segmentation tools usually adopt three major schemes: the FeTA 7-label scheme, the dHCP atlas-based protocols (9-, 17- and 91-label schemes), and the BOUNTI pipeline with 19 labels.

The FeTA protocol segments the fetal brain into seven broad anatomical labels: cerebrospinal fluid, cortical grey matter, white matter, ventricles, cerebellum, deep grey matter and brainstem\,\cite{Payette2021,FeTA2024_paper}. This scheme was inspired by neonatal segmentation in the dHCP project, and targets tissue compartments that are important for detecting developmental abnormalities\,\cite{Payette2021}. Its simplicity enhances robustness across varying MRI conditions and gestational ages. It is suitable for benchmarking segmentation models and clinical volumetry tasks. However, it lacks anatomical granularity and does not delineate subcortical structures individually.

The developing Human Connectome Project (dHCP) provides fetal parcellations of 9, 17 and 91 labels. The main strength is anatomical coverage combined with publicly available age-specific templates and segmentation maps. The 17-label protocol is more detailed than FeTA while still feasible for segmentation. For many structures a left-right distinction is made. However, extremely small or transient structures are not included.

\textsc{bounti} (Brain vOlumetry and aUtomated parcellatioN for 3D feTal MRI)\,\cite{Uus2023} is a deep learning pipeline trained on dHCP-derived manual labels, producing 19-region segmentations. It builds on the dHCP atlas and refines it with clinical feedback. Labels were selected for their anatomical relevance, MRI visibility, and clinical importance\,\cite{Uus2023}. Its 19-label definition allows fine-grained volumetric analysis. However, it lacks subdivisions into standard anatomical regions---e.g., the frontal lobe.

As already mentioned, no single protocol has become the universal standard for fetal brain segmentation. FeTA parcellation is popular for algorithm development and benchmarking, thanks to the public datasets released in conjunction with the editions of the FeTA Challenge. dHCP and \textsc{bounti} are favored for anatomical accuracy, with dHCP also providing a large, high-quality public dataset.

\section{Super-Resolution Reconstruction} \label{sec:SuperResolutionReconstruction}
Super-resolution reconstruction (SRR) addresses the intrinsic anisotropy and motion corruption that characterize in-utero T2-weighted single-shot acquisitions by fusing multiple, misaligned 2D stacks into a motion-corrected, isotropic 3D volume. Fetal SRR is commonly formulated as an inverse problem, which involves finding the original scene that generates the acquired images under the imaging conditions\,\cite{Gholipour2010}. A slice acquisition model is needed, that should be robust to motion-corrupted and mis-registered slices, and noise. Early robust estimators for bias field handling established the methodological backbone. Subsequent developments delivered total variation and edge-preserving regularization, to prevent amplification of noise and registration error\,\cite{Kuklisova2012}.

Given multiple stacks of thick slices acquired in approximately orthogonal planes, SRR models each observed slice as a blurred and resampled version of an unknown high-resolution volume. Only three modern toolkits provide end-to-end functionality---brain localization, robust SRR, and standardized-space alignment---and constitute the framework for fetal brain SRR\,\cite{Ciceri2023_SRR}: NiftyMIC\,\cite{Ebner2020}, \textsc{mialsrtk}\,\cite{Tourbier2015}, and \textsc{simple irtk}\,\cite{Kuklisova2012}.

An intensity-matching SRR with slice-wise bias handling and with rigid SVR was originally implemented in the \textsc{simple irtk} framework\,\cite{irtk-simple}. It demonstrated high-quality reconstructions across challenging gestational ages and limited data, while highlighting the necessity of robust outlier rejection\,\cite{Kuklisova2012}.

Building on the above, Tourbier and colleagues\,\cite{Tourbier2015} implemented a total variation-regularized SRR, which has been extensively investigated due to its capacity to preserve tissue interfaces while controlling noise. The pipeline employs a fast convex optimization technique for SRR with adaptive regularization. The resulting \textsc{mialsrtk} toolbox\,\cite{MIALSRTK} established as a practical choice for fetal reconstructions, thanks to its resilience to motion and residual misregistrations.

NiftyMIC provides a fully automated pipeline coupling deep-learning-based brain localization and segmentation with an outlier-robust SRR and standardized template-space alignment. Controlled experiments recommend the acquisition of at least three approximately orthogonal stacks---preferably five stacks across three orientations---to ensure sufficient angular sampling and partial-volume recovery\,\cite{Ebner2020}. The framework has been further adapted to fetal fMRI, with Huber L2 regularization and reference-volume motion correction\,\cite{Sobotka2022}.

Recent studies have assessed SRR reliability and method-specific artifacts across the main reconstruction pipelines. A multi-rater quality assessment study\,\cite{Sanchez2024a} reported consistent quality scores and analyzed typical reconstruction artifacts among different toolkits. The results show that \enquote{excellent reliability can be achieved for global quality scoring across three raters, with good reliability on specific criteria relating to the contrast across tissues and noise levels}\,\cite{Sanchez2024a}. A complementary multi-centric biometry and volumetry study\,\cite{Sanchez2024b} reconstructed each case with multiple SRR methods to test the consistency of biometric and volumetric measurements. The paper confirms that SRR methods don't alter the studied measurements, which would remain consistent across sites. However, the authors warn about potential intensity alterations and biases across centers and scanners.

\section{Challenges} \label{sec:Challenges}
MRI, and in particular fetal brain MRI, presents some characteristics that contribute to significant domain shifts, compromising the performance of deep learning models on unseen data distributions\,\cite{FeTA2024_paper, FeTA2024_review}.

On the biological aspect, the fetal brain is very challenging \textit{per se}, since it undergoes rapid and complex changes throughout gestation:
\begin{description}
    \item[Rapid Morphological Evolution] The brain structure and appearance reorganize dramatically during prenatal development\,\cite{FeTA2024_paper, Ciceri2023_SEG}. Defining and consistently identifying different brain structures across varying gestational ages is difficult due to ongoing neuronal migration, gyrification, and sulcation patterns. This means that images acquired at different GAs constitute distinct sub-domains, making a single model challenging to generalize across the entire gestational spectrum\,\cite{Ciceri2024}.
    \item[Low Tissue Contrast] The fetal brain shows different tissue contrast compared to postnatal brains. Specifically, the intensity difference between white matter and gray matter is reduced due to the absence of myelin in the fetal WM. This causes WM to appear brighter than GM on T2w images\,\cite{Ciceri2023_SEG}. Tissue contrast changes are also significantly influenced by gestational age\,\cite{FeTA2024_paper}.
    \item[Pathological Heterogeneity] Congenital disorders introduce further significant morphological variations. Training segmentation algorithms exclusively on neurotypical samples can reduce their robustness when encountering altered morphologies typical of pathological cases (e.g., spina bifida, ventriculomegaly, corpus callosum malformations). The rarity and wide variability of fetal pathologies contribute to the challenge of obtaining sufficient data for exhaustive datasets\,\cite{FeTA2024_paper, FeTA2024_review}.
\end{description}

On the technical side, the nature of \textit{in-utero} MRI introduces further limitations\,\cite{Ciceri2023_SEG}:
\begin{description}
    \item[Motion Artifacts] Spontaneous fetal movement and maternal breathing during MRI acquisition lead to various artifacts, such as in-plane image blur, slice crosstalk and incongruence in slice location, significantly degrading the quality of individual slices and the reconstructed volume. While ultrafast 2D sequences (e.g., \textsc{ssfse}, \textsc{haste}) are commonly employed to minimize these effects by acquiring each slice rapidly, they cannot entirely eliminate motion-related issues.
    \item[Low Contrast-to-Noise Ratio] The small size of the fetal brain and the need for shorter scanning periods to minimize motion contribute to a low CNR. This, coupled with acquisition limits such as thick slices---to achieve good SNR--- makes it difficult to distinguish fine anatomical details.
    \item[Intensity Inhomogeneities] Radiofrequency field inhomogeneity, non-uniform reception coil sensitivity, eddy currents driven by field gradients, and electromagnetic interactions with the body can cause intensity inhomogeneities. Such variations produce intensity bias fields across the image space, making consistent tissue intensity representation challenging for automated methods.
    \item[Partial Volume Effects] Due to thick slices, a single voxel may contain signals from multiple tissue types. This \enquote{mixing} of signals leads to ambiguous boundaries and can result in mislabeled segmentations, especially at tissue interfaces. For instance, the mixing of the CSF and cortical GM boundary leads to intensities similar to the intensity profile of the WM.
\end{description}

Ultimately, the scarcity of large, standardized, and annotated datasets further jeopardizes the development of robust segmentation algorithms.\,\cite{FeTA2024_paper,FeTA2021_review, Ciceri2023_SEG} This is due to:
\begin{description}
    \item[Small and Heterogeneous Cohorts] Fetal MRI studies often require collaboration from specialized clinical centers due to the small and vulnerable patient populations. This leads to small datasets at individual institutions, making it difficult to create uniform in-house single-center datasets.
    \item[Variability in Acquisition Protocols and Hardware] Data from different clinical centers tend to exhibit considerable variability in image acquisition parameters, MRI scanner hardware, and manufacturer specifications. The most important differences use to be in magnetic field strengths (e.g., \qtylist{0.55;1.5;3}{\tesla}) and specific sequence parameters (e.g., TR/TE values, flip angles, FOV, slice thickness). The associated acquisition shift cause deep learning models trained on one domain to perform poorly on data from another.
    \item[SR Reconstruction Variability] Fetal MR images are typically acquired as orthogonal stacks of 2D slices to mitigate motion, which then require SR reconstruction algorithms to generate high-resolution 3D volumes. The use of different algorithms (e.g., \textsc{mialsrtk}\,\cite{Tourbier2015, MIALSRTK}, \textsc{irtk}\,\cite{Kuklisova2012, irtk-simple}, \textsc{NiftyMIC}\,\cite{Ebner2020}, \textsc{svrtk}\,\cite{Uus2022}) introduces an additional layer of domain shift.
    \item[Manual Annotation Variability] Creating high-quality, manually labeled data is time-consuming---often taking several hours per case---and requires expert anatomical knowledge. This process is also prone to human error and significant inter-rater variability.
\end{description}

In conclusion, the multifaceted nature of these challenges makes robust and generalizable fetal brain MRI segmentation an exceptionally difficult task.
