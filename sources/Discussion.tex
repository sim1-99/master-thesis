%%%%%%%%%%%%%%%%%%%%%%%%%%%%%%%%%%%%%%%%%%%%%%%%%%%%%%%%%%%%%%%%%%%%%%%%
\chapter{Discussion} \label{chap:Discussion}
%%%%%%%%%%%%%%%%%%%%%%%%%%%%%%%%%%%%%%%%%%%%%%%%%%%%%%%%%%%%%%%%%%%%%%%%
\vspace{1cm}

This chapter discusses the empirical evidence reported in Chapter \ref{chap:Results}, drawing a unified interpretation across general performance trends across datasets, pairwise model comparisons, and pathology-stratified analyses. The emphasis is on the behavior of the augmentation strategies in realistic in-domain and out-of-domain settings, and on the conditions under which gains are observed.

\section{Primacy of Data over Augmentation}
The first, and most consistent, observation is the primacy of the training dataset over the augmentation recipe. Models trained on dHCP achieve the most stable performance both ID and OOD, and do so \emph{independently} of whether augmentation is the nnU-Net default, GIN-IPA, or their combination. In Section \ref{sec:GeneralPerformance}, this is visible in the DSC histograms (Figs.\,\ref{fig:default_DC}--\ref{fig:both_DC}): while Kispi-trained models experience a marked OOD drop---particularly towards dHCP, with ventricles, dGM and WM most affected---the dHCP-trained models exhibit limited degradation across inference domains. Moreover, no DSC gain emerges from switching augmentation when training on dHCP. Together, these findings imply that \emph{dataset quality and scale} dominate generalization in this setting.

A further confirmation of the importance of the data quality emerges for Kispi-mial: the change of domain does not impact the network trained on Kispi-mial in the same manner as the other two sources. This is partially attributable to the lower image quality in Kispi-mial \cite{FeTA2021_review}. It is therefore expected that any model evaluated OOD on Kispi-mial will underperform relative to other datasets.

\section{Efficacy of GIN-IPA}
\paragraph{Baseline vs.\ GIN-IPA.}
When training on Kispi-irtk and inferring on dHCP, GIN-IPA yields a \emph{substantial improvement} over the baseline across metrics and labels. At the global level, the average DSC increases (\qty{+58}{\percent}), with concordant improvements in VS (\qty{-36}{\percent}) and HD95(\qty{-22}{\percent}). The corresponding paired analysis confirms the statistical significance and a large effect size for DSC in this cross-domain transfer (DSC: \numrange[range-open-phrase=from\ ]{0.34}{0.54}, $p \ll 0.01$, $|d|=1.1$; VS: \numrange[range-open-phrase=from\ ]{1.12}{0.71}, $p\ll 0.01$, $|d|=0.8$; HD95: \numrange[range-open-phrase=from\ ]{49}{38}, $p\ll 0.01$, $|d|=0.6$; complete data are in Appendix \hyperref[app:SupplementaryTables]{B}).

Conversely, training on dHCP shows no material benefit from GIN-IPA over the baseline across any inference domain or label. For Kispi-mial, effects are small and structure-dependent, with modest improvements predominantly in cases where the baseline struggles on poorly segmented volumes.

\paragraph{GIN-IPA vs.\ Combined (default + GIN-IPA).}
Stacking the two augmentation methods does not systematically improve performance over GIN-IPA alone. When trained on Kispi-irtk and inferred on dHCP, the combined strategy is consistently inferior to pure GIN-IPA for several structures (CSF, cerebellum, dGM) across all metrics. Training on dHCP again yields no differences between the two.

\section{Robustness by Pathology}
The stratified analysis (Section \ref{sec:PerformanceByPathology}) indicates that models trained on dHCP generalize equally well to healthy subjects in Kispi-mial and Kispi-irtk, despite the domain shift (different image quality and reconstruction techniques). Performance decreases for pathological cases, reflecting segmentation challenges in the presence of anatomical abnormalities. In this more difficult regime, GIN-IPA achieves very small improvements over the baseline, while remaining broadly equivalent to the combined strategy. These results also suggest that the lower-quality images are concentrated among pathological scans, explaining the larger performance gap in that subgroup.

\section{General Outcomes}
The evidence above supports three general conclusions:
\begin{itemize}
    \item \textbf{Data quality and scale prevail:} When training data are abundant and homogeneous (like in dHCP), model choice of augmentation among those tested barely alters performance. The converse is also true: with smaller, noisier sources (such as Kispi), OOD degradation is pronounced.
    \item \textbf{GIN-IPA is conditionally beneficial:} Its gains are largest in the domain generalization scenario from Kispi-irtk to dHCP, where it closes a considerable part of the OOD gap. Effects shrink or vanish when the source dataset already captures the target variability.
    \item \textbf{Stacking augmentations is not additive:} The combined strategy often overlaps with, and can even dilute, the benefits of GIN-IPA; it never consistently outperforms GIN-IPA in the examined cross-domain settings.
\end{itemize}

These patterns are coherent with the intended role of GIN-IPA: by synthesizing intensity and spatial variations, it exposes the network to harder, more diverse views of a limited source, which is most valuable when the source lacks the target domain variability. Once data already cover the relevant distributional modes (as with dHCP), marginal augmentation gains become negligible.

\section{Limitations and Future Work}
Three principal aspects delimit the scope of the present findings. First, the small size of the Kispi cohorts constrained further stratification (e.g., finer gestational age, training by pathology) and limits the precision of subgroup estimates. Generally speaking, the limited availability of public datasets, together with their high intra-variability, jeopardizes the possibility to totally isolate single domains. Second, label-set harmonization between Kispi and dHCP required a mapping (Tab.\,\ref{tab:label_merge}), which, although carefully defined, introduces an additional layer of variability in label-wise comparisons. Third, because of hardware limitations, the IPA step was performed in its 2D variant, which may be less effective than the full 3D version.

Practically, the results argue for prioritizing \emph{data curation}---larger, multi-center, high-quality fetal MRI with standardized SRR pipelines---since source quality and scale are the dominant predictors of cross-domain success in this task. Within constrained-data regimes, GIN-IPA is a \emph{useful augmentation choice}, particularly for single-source DG from moderately sized, relatively clean sources (e.g., Kispi-irtk). Nonetheless, stacking it with standard nnU-Net augmentation is unnecessary and sometimes counterproductive. For challenging pathological cases, improved acquisition and reconstruction remain central to performance gains.

Besides data, it could be interesting to disentangle the two components of GIN-IPA---possibly employing the 3D IPA variant---to assess their individual contributions, and to explore other augmentation strategies (e.g., adversarial, style-transfer) in this setting.
