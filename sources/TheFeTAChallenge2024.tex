%%%%%%%%%%%%%%%%%%%%%%%%%%%%%%%%%%%%%%%%%%%%%%%%%%%%%%%%%%%%%%%%%%%%%%%%
\chapter{The FeTA Challenge 2024}
%%%%%%%%%%%%%%%%%%%%%%%%%%%%%%%%%%%%%%%%%%%%%%%%%%%%%%%%%%%%%%%%%%%%%%%%
\vspace{1cm}

The Fetal Tissue Annotation Challenge (\textsc{FeTA}) \cite{FeTA2024} was born in 2020, and joined the International Conference on Medical Image Computing and Computer-Assisted Intervention (MICCAI) \cite{MICCAI} in 2021. Up to now, four editions have been organized (in 2020, 2021, 2022, and 2023), with increasing participation and interest from the medical imaging community. The main contributions of the \textsc{FeTA} challenge are the creation of a benchmark dataset for fetal brain MRI segmentation and biometry, and the promotion of the development of algorithms for the automatic segmentation of fetal brain tissues.

\textsc{FeTA} 2024 is divided into two sub-challenges.
\begin{description}
\item[Generalizable Fetal Brain Segmentation] The fist task is about multi-class segmentation, where participants are asked to segment the fetal brain into seven different tissues: cerebrospinal fluid (CSF), cortical gray matter (GM), white matter (WM), lateral ventricles (LV), deep gray matter (DGM), brainstem (BS), and cerebellum (CB). The performance is evaluated using different metrics, such as the Dice similarity coefficient between the predicted and ground truth segmentations.
\item[Fetal Brain Biometry] The second task deals with the estimation of five different biometric measurements of fetal brain anatomical structures: length of the corpus callosum (LCC), height of the vermis (HV), brain biparietal diameter (bBIP\_ax), skull biparietal diameter (sBIP\_ax), and maximum transverse cerebellar diameter (TCD\_cor). In this case, the results are evaluated using the relative error between the predicted and ground truth measurements. \cite{FeTA2024_paper}
\end{description}

\section{Previous Editions}
The first edition of the \textsc{FeTA} challenge was organized in 2020. It was mainly focused 

\section{Fetal Brain Segmentation}

\section{Fetal Brain Biometry}
