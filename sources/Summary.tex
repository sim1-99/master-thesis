\addcontentsline{toc}{part}{Summary}
%%%%%%%%%%%%%%%%%%%%%%%%%%%%%%%%%%%%%%%%%%%%%%%%%%%%%%%%%%%%%%%%%%%%%%%%
\chapter*{Summary}
%%%%%%%%%%%%%%%%%%%%%%%%%%%%%%%%%%%%%%%%%%%%%%%%%%%%%%%%%%%%%%%%%%%%%%%%
\vspace{1cm}

This thesis originates from my internship at Universitat Pompeu Fabra (Barcelona, Spain), within the Erasmus+ Traineeship program. Here, I participated in the FeTA Challenge 2024, with the active contribution and support of Dr.\ Gerard Martí Juan, and under the supervision of Prof.\ Miguel Ángel González Ballester. Once back to Bologna, I continued my research employing the knowledge and tools acquired during the internship, under the guidance of Prof.\ Daniel Remondini and Dr.\ Nico Curti.

Accurate segmentation of fetal brain tissues from magnetic resonance imaging is essential to study early neurodevelopment. However, this task remains challenging due to the large variability introduced by different acquisition settings, scanner hardware, and super-resolution reconstruction pipelines. These differences produce distinct imaging domains, which can lead segmentation models to perform inconsistently when applied to data that differ from those used during training. This thesis investigates whether a 3D U-Net can be made more robust to such variability through the use of an augmentation method, GIN-IPA, originally designed for single-source domain generalization.

The study considers three fetal MRI datasets—Kispi-mial, Kispi-irtk, and dHCP—which differ in size, reconstruction method, and image quality. All data were harmonized to a common seven-tissue parcellation to allow consistent training and evaluation. As baseline architecture, the nnU-Net framework was employed in its 3D full-resolution configuration. Three data augmentation strategies were compared: (i) the default nnU-Net augmentation pipeline, (ii) GIN-IPA used as an augmentation method, and (iii) a combined strategy applying both. Each model was trained separately on each dataset and evaluated both within the same domain and on the remaining datasets. Performance assessment relied on the metrics used in the FeTA Challenge.

The results indicate that the characteristics of the training dataset play a central role in determining the stability of the segmentation across domains. Models trained on the larger and more homogeneous dHCP dataset show consistent performance on all test domains, with minimal dependence on the augmentation strategy. In contrast, models trained on the smaller Kispi datasets exhibit a stronger sensitivity to domain changes. In these cases, GIN-IPA leads to visibly improved robustness compared to the default augmentation. However, combining GIN-IPA with the standard augmentation pipeline does not yield further benefits and may reduce stability in some settings.

Overall, the findings show that high-quality training data remain the most effective means to ensure reliable cross-domain performance in fetal brain MRI segmentation. Within more restricted training scenarios, GIN-IPA represents a practical augmentation option that can improve robustness to acquisition-related variability.

The work is organized in three parts and eight chapters:
\begin{description}
    \item[Part I.\ Introduction] Comprises Chapters 1-3, establishing the theoretical and technical background of the thesis. It introduces the physical principles underlying magnetic resonance imaging, outlines the characteristics of fetal brain imaging, and reviews the state-of-the-art deep learning segmentation methods with a particular focus on domain generalization. The Part concludes by presenting the context of the FeTA Challenge, which serves as the field benchmark.
    \item[Part II.\ Materials and Methods] Comprises Chapters 4-6, presenting the datasets, the methodological framework, and the experimental workflow. Specifically, the acquisition and reconstruction features of the Kispi and dHCP datasets are detailed, as well as the nnU-Net architecture and the GIN-IPA augmentation strategy, and the training configuration. It also presents the evaluation metrics and statistical tools employed.\item[Part III.\ Results and Discussion] Comprises Chapters 7 and 8, which synthesizes the experimental results and discusses their broader implications. It highlights the dominant role of data quality and scale in determining cross-domain performance, and evaluates the effectiveness of GIN-IPA. Finally, limitations of the present study and directions for future research are mentioned.
\end{description}
